%%%%%%%%%%%%%%%%%%%%%%%%%%%%%%%%%%%%%%%%%%%%%%%%%%%%%%%%%%%%%%%%%%%%%%%%%%%%%%%%
%2345678901234567890123456789012345678901234567890123456789012345678901234567890
%        1         2         3         4         5         6         7         8

\documentclass[letterpaper, 10 pt, conference]{ieeeconf}  % Comment this line out if you need a4paper

%\documentclass[a4paper, 10pt, conference]{ieeeconf}      % Use this line for a4 paper

\IEEEoverridecommandlockouts                              % This command is only needed if 
                                                          % you want to use the \thanks command

\overrideIEEEmargins                                      % Needed to meet printer requirements.

%In case you encounter the following error:
%Error 1010 The PDF file may be corrupt (unable to open PDF file) OR
%Error 1000 An error occurred while parsing a contents stream. Unable to analyze the PDF file.
%This is a known problem with pdfLaTeX conversion filter. The file cannot be opened with acrobat reader
%Please use one of the alternatives below to circumvent this error by uncommenting one or the other
%\pdfobjcompresslevel=0
%\pdfminorversion=4

% See the \addtolength command later in the file to balance the column lengths
% on the last page of the document

% The following packages can be found on http:\\www.ctan.org
%\usepackage{graphics} % for pdf, bitmapped graphics files
%\usepackage{epsfig} % for postscript graphics files
%\usepackage{mathptmx} % assumes new font selection scheme installed
%\usepackage{times} % assumes new font selection scheme installed
%\usepackage{amsmath} % assumes amsmath package installed
%\usepackage{amssymb}  % assumes amsmath package installed
\usepackage{outlines}

\title{\LARGE \bf
Draft Title: Handheld Haptic Guidance Device
}  % or: Learning How To Teleoperate Humans


\author{Albert Author$^{1}$ and Bernard D. Researcher$^{2}$% <-this % stops a space
\thanks{*This work supported by...}% <-this % stops a space
\thanks{$^{1}$Albert Author is with Faculty of Electrical Engineering, Mathematics and Computer Science,
        University of Twente, 7500 AE Enschede, The Netherlands
        {\tt\small albert.author@papercept.net}}%
\thanks{$^{2}$Bernard D. Researcheris with the Department of Electrical Engineering, Wright State University,
        Dayton, OH 45435, USA
        {\tt\small b.d.researcher@ieee.org}}%
}


\begin{document}



\maketitle
\thispagestyle{empty}
\pagestyle{empty}


%%%%%%%%%%%%%%%%%%%%%%%%%%%%%%%%%%%%%%%%%%%%%%%%%%%%%%%%%%%%%%%%%%%%%%%%%%%%%%%%
\begin{abstract}

We made a haptic device. We tried it on people and modeled how they moved. We used MPC to build a controller based on the user model to guide people through trajectories. We found that it improved performance.

\end{abstract}


%%%%%%%%%%%%%%%%%%%%%%%%%%%%%%%%%%%%%%%%%%%%%%%%%%%%%%%%%%%%%%%%%%%%%%%%%%%%%%%%
\section{Introduction}

\begin{outline}
    %  \1 Benefits of haptics in surgical training
    %     \2 Cost of learning is high
    %     \2 Haptics can speed learning (cite) 
    \1 Handheld haptic devices
        \2 benefits: can be mobile, large workspace, easy donn and doff
        \2 current mobile, large workspace devices typically rely on simple haptic effects (vibration)
        \2 Can distribute reaction forces across the palm through a handle and provide kinesthetic forces to fingertips. 
    \1 Extra challenges:
        \2 No net forces and torques.
        \2 Can't completely prescribe or restrict user's movements 
        \2 Must be dependent on the users' response
    \1 Other research and the limitations:
        \2 Wearable fingertip devices 
            \3 lack real kinesthetic forces on fingerpads
            \3 difficult donning and doffing 
            \3 Pacchierotti review paper\cite{Pacchierotti2017} (update vol., page num)
        \2 Grippers: no fingertip tangential forces, (why is tangential skin deformation important)
            \3 Kuchenbecker, Pierce teleoperation gripper \cite{Pierce2014}
            \3 CLAW \cite{Choi2018} (check page num)
            \3 Wolverine? Counts as holable?
            \3 Grabbity \cite{Choi2017}
        \2 Palm based force devices, not suitable for fine manipulation tasks like surgical training
            \3 \cite{Walker2017, Yano2003}torque feedback 
            \3 Directional handle from CHI paper (published yet?)
            \3 Provancher skin stretch handle \cite{Gleeson2009}
            \3 Haptic links \cite{Strasnick2018} (check page num)
            \3 Air Wand \cite{Romano2009}
        \2 Texture rendering: lack fingertip force information
            \3 haptic revolver \cite{Whitmire2018}
            \3 Haptic pen (Culbertson) (CITE)
            \3 CLAW

    \1 Our device 
        \2 A 2-dof pantograph mechanism on the index and thumb provide tangential skin-deformation 
        \2 At the hinge between the thumb abd index finger, a motor provides grip resistence 
        \2 We demonstrate that users can be guided to move in two translation and two rotation directions
        \2 In a demo application, users completed a needle passing task with haptic guidance from the device (maybe)

\end{outline}


\section{Device}

\subsection{Design}
\begin{outline}
\1 Two pantagraph mechanisms providing 2DOF kinesthetic forces to fingertips
\1 Reaction forces on palm at handle

\end{outline}

\subsection{System}
\begin{outline}
\1 Ascension TRAKstar magnetic tracking system records position and orientation of handle at XXX Hz
\1 Magnetic tracking and motor actuation done through a Visual Studio program with a Sensoray 826 
\1 Modeling and MPC (or MCTS) implemented in Julia. Using Ipopt Optimization package.
\1 Visual Studio program and Julia optimal control communicate through ZMQ
\end{outline}

\section{Validation}
\subsection{Simulations}
\subsection{User Study}
\input{F_discussion.tex}


\begin{table}[h]
\caption{An Example of a Table}
\label{table_example}
\begin{center}
\begin{tabular}{|c||c|}
\hline
One & Two\\
\hline
Three & Four\\
\hline
\end{tabular}
\end{center}
\end{table}


   \begin{figure}[thpb]
      \centering
      \framebox{\parbox{3in}{We suggest that you use a text box to insert a graphic (which is ideally a 300 dpi TIFF or EPS file, with all fonts embedded) because, in an document, this method is somewhat more stable than directly inserting a picture.
}}
      %\includegraphics[scale=1.0]{figurefile}
      \caption{Example Figure: Inductance of oscillation winding on amorphous
       magnetic core versus DC bias magnetic field}
      \label{figurelabel}
   \end{figure}
   

Figure Labels: Use 8 point Times New Roman for Figure labels. Use words rather than symbols or abbreviations when writing Figure axis labels to avoid confusing the reader. As an example, write the quantity �Magnetization�, or �Magnetization, M�, not just �M�. If including units in the label, present them within parentheses. Do not label axes only with units. In the example, write �Magnetization (A/m)� or �Magnetization {A[m(1)]}�, not just �A/m�. Do not label axes with a ratio of quantities and units. For example, write �Temperature (K)�, not �Temperature/K.�


\addtolength{\textheight}{-12cm}   % This command serves to balance the column lengths
                                  % on the last page of the document manually. It shortens
                                  % the textheight of the last page by a suitable amount.
                                  % This command does not take effect until the next page
                                  % so it should come on the page before the last. Make
                                  % sure that you do not shorten the textheight too much.

%%%%%%%%%%%%%%%%%%%%%%%%%%%%%%%%%%%%%%%%%%%%%%%%%%%%%%%%%%%%%%%%%%%%%%%%%%%%%%%%



%%%%%%%%%%%%%%%%%%%%%%%%%%%%%%%%%%%%%%%%%%%%%%%%%%%%%%%%%%%%%%%%%%%%%%%%%%%%%%%%



%%%%%%%%%%%%%%%%%%%%%%%%%%%%%%%%%%%%%%%%%%%%%%%%%%%%%%%%%%%%%%%%%%%%%%%%%%%%%%%%
\section*{Appendix}

Appendixes should appear before the acknowledgment.

\section*{ACKNOWLEDGMENT}

The preferred spelling of the word �acknowledgment� in America is without an �e� after the �g�. Avoid the stilted expression, �One of us (R. B. G.) thanks . . .�  Instead, try �R. B. G. thanks�. Put sponsor acknowledgments in the unnumbered footnote on the first page.



%%%%%%%%%%%%%%%%%%%%%%%%%%%%%%%%%%%%%%%%%%%%%%%%%%%%%%%%%%%%%%%%%%%%%%%%%%%%%%%%

References are important to the reader; therefore, each citation must be complete and correct. If at all possible, references should be commonly available publications.



\bibliography{WHC2019.bib}{}
\bibliographystyle{IEEEtran}



\end{document}
