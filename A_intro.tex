\section{Introduction}

\begin{outline}
    %  \1 Benefits of haptics in surgical training
    %     \2 Cost of learning is high
    %     \2 Haptics can speed learning (cite) 
    \1 Handheld haptic devices
        \2 benefits: can be mobile, large workspace, easy donn and doff
        \2 current mobile, large workspace devices typically rely on simple haptic effects (vibration)
        \2 Can distribute reaction forces across the palm through a handle and provide kinesthetic forces to fingertips. 
    \1 Extra challenges:
        \2 No net forces and torques.
        \2 Can't completely prescribe or restrict user's movements 
        \2 Must be dependent on the users' response
    \1 Other research and the limitations:
        \2 Wearable fingertip devices 
            \3 lack real kinesthetic forces on fingerpads
            \3 difficult donning and doffing 
            \3 Pacchierotti review paper\cite{Pacchierotti2017} (update vol., page num)
        \2 Grippers: no fingertip tangential forces, (why is tangential skin deformation important)
            \3 Kuchenbecker, Pierce teleoperation gripper \cite{Pierce2014}
            \3 CLAW \cite{Choi2018} (check page num)
            \3 Wolverine? Counts as holable?
            \3 Grabbity \cite{Choi2017}
        \2 Palm based force devices, not suitable for fine manipulation tasks like surgical training
            \3 \cite{Walker2017, Yano2003}torque feedback 
            \3 Directional handle from CHI paper (published yet?)
            \3 Provancher skin stretch handle \cite{Gleeson2009}
            \3 Haptic links \cite{Strasnick2018} (check page num)
            \3 Air Wand \cite{Romano2009}
        \2 Texture rendering: lack fingertip force information
            \3 haptic revolver \cite{Whitmire2018}
            \3 Haptic pen (Culbertson) (CITE)
            \3 CLAW

    \1 Our device 
        \2 A 2-dof pantograph mechanism on the index and thumb provide tangential skin-deformation 
        \2 At the hinge between the thumb abd index finger, a motor provides grip resistence 
        \2 We demonstrate that users can be guided to move in two translation and two rotation directions
        \2 In a demo application, users completed a needle passing task with haptic guidance from the device (maybe)

\end{outline}

