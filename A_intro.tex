\section{Introduction}

\begin{outline}
    %  \1 Benefits of haptics in surgical training
    %     \2 Cost of learning is high
    %     \2 Haptics can speed learning (cite) 
    \1 Handheld haptic devices
        \2 benefits: can be mobile, large workspace, easy don and doffing
        \2 current mobile, large workspace devices typically rely on simple haptic effects (vibration)
    \1 What are the extra challenges?
        \2 No net forces and torques.
        \2 Can't completely prescribe or restrict user's movements 
    \1 What have other researchers done, and what are its limitations?
        \2 Grippers: no fingertip tangential forces, (why is tangential skin deformation important)
            \3 Kuchenbecker, Pierce teleoperation gripper \cite{Pierce201}
            \3 CLAW \cite{Choi2018} (check page num)
            \3 Wolverine? Counts as holable?
        \2 Palm based force devices, not suitable for fine manipulation tasks like surgical training
            \3 \cite{Walker2017} torque feedback 
            \3 Directional handle in CHI paper
            \3 Provancher skin stretch handle \cite{Gleeson2009}
            \3 Haptic links \cite{Strasnick2018} (check page num)
        \2 Texture rendering: lack fingertip force information
            \3 haptic revolver \cite{Whitmire2018}
            \3 Haptic pen (Culbertson)
            \3 CLAW
        \2 Wearable fingertip devices (Prattichizo, Schorr), but don't provide kinesthetic resistance to grip force, donning/doffing takes effort \cite{Pacchierotti2017} (update vol., page num)
    \1 Our device 
        \2 A 2-dof pantograph mechanism on the index and thumb provide tangential skin-deformation 
        \2 At the hinge between the thumb abd index finger, a motor provides grip resistence 
        \2 We demonstrate that users can be guided to move in two translation and two rotation directions
        \2 In a demo application, users completed a needle passing task with haptic guidance from the device (maybe)

\end{outline}

